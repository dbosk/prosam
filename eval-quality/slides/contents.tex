\mode*

\section{Uppvärmning/check-in}

\begin{frame}
  \begin{question}[Helgrupp, 10 minuter]
    \begin{enumerate}
      \item Var befinner du dig just nu?
      \item Var skulle du vilja befinna dig just nu?
    \end{enumerate}
  \end{question}
\end{frame}


\section{Återkoppling}

\begin{frame}
  \begin{question}[Breakout rooms, 10 minuter]
    \begin{itemize}
      \item Vilken typ av studentåterkoppling är användbar för att en 
        kursledare ska kunna göra en kurs bättre?
    \end{itemize}
  \end{question}
\end{frame}

\begin{frame}
  \begin{remark}[Helgrupp, 10 minuter]
    \begin{itemize}
      \item Vilka aspekter på kvalitet kan studenter utvärdera själva och 
        således ge (korrekt) återkoppling på?
      \item Exempelvis 
        \citetitle{MeasuringActualLearningVSFeeling}\footfullcite{MeasuringActualLearningVSFeeling}.
      \item Eller valfri kurs i användarcentrerad utveckling:
        \citetitle{FirstRuleDontListenToUsers}\footfullcite{FirstRuleDontListenToUsers}.
    \end{itemize}
  \end{remark}
\end{frame}


\section{Kvalitet på kurser}

\begin{frame}
  \begin{block}{10 kriterier från tidigare prosam}
    \begin{enumerate}
      \item Kunniga och engagerade lärare
      \item Bra kursupplägg och litteratur
      \item Studentaktiverande undervisning
      \item Lärande examination (förtydligande: examination som inte bara är 
        till för att sätta betyg utan som man lär sig av också, t ex många 
        labbar)
      \item Bra information och kursadministration
      \item Kurser som passar bra ihop
      \item Bra fysisk studiemiljö
      \item Relevanta kurser för utbildningen, plats för valfrihet
      \item Bra stöd för den som behöver
      \item Bra studiekamrater
    \end{enumerate}
  \end{block}
\end{frame}

\begin{frame}
  \begin{block}{10 kriterier från tidigare prosam}
    \begin{enumerate}
      \item Kunniga och engagerade lärare
      \item Bra kursupplägg och litteratur
      \item Studentaktiverande undervisning
      \item Lärande examination (förtydligande: examination som inte bara är 
        till för att sätta betyg utan som man lär sig av också, t ex många 
        labbar)
      \item Bra information och kursadministration
        \saveenumi
    \end{enumerate}
  \end{block}

  \begin{question}[Helgrupp, 5 minuter]
    \begin{itemize}
      \item Hur kan vem utvärdera huruvida dessa kriterier är uppfyllda?
    \end{itemize}
  \end{question}
\end{frame}

\begin{frame}
  \begin{question}[Helgrupp, 5 minuter]
    \begin{itemize}
      \item Hur kan vem utvärdera huruvida dessa kriterier är uppfyllda?
    \end{itemize}
  \end{question}

  \begin{block}{10 kriterier från tidigare prosam}
    \begin{enumerate}\resumeenumi
      \item Kurser som passar bra ihop
      \item Bra fysisk studiemiljö
      \item Relevanta kurser för utbildningen, plats för valfrihet
      \item Bra stöd för den som behöver
      \item Bra studiekamrater
    \end{enumerate}
  \end{block}
\end{frame}

%\subsection{Kvalitet på datakurser}

\begin{frame}
  \begin{exercise}[Breakout rooms, 10 minuter]
    \begin{itemize}
      \item Välj en datakurs från varje årskurs och jämför kursernas kvalitet 
        (t ex med kriterierna 1-5).
    \end{itemize}
  \end{exercise}
\end{frame}

%\subsection{Kvalitet på komplementära kurser}

\begin{frame}
  \begin{exercise}[Breakout rooms, 10 minuter]
    \begin{itemize}
      \item Välj en teknikkomplementär kurs (inte datalogi eller matematik) 
        från varje årskurs och jämför kursernas kvalitet.
    \end{itemize}
  \end{exercise}
\end{frame}

