\mode*

\begin{frame}
  \begin{question}
    \begin{itemize}
      \item Vad kan man göra i ett grupparbete om någon prokrastinerar sitt 
        arbete?
    \end{itemize}
  \end{question}
\end{frame}

\begin{frame}
  \begin{question}
    \begin{itemize}
      \item Är mer uppstyckad examination med många regelbundna deadlines en 
        bra lösning för alla kurser? Vad finns det för andra upplägg som kan 
        motverka prokrastinering?
    \end{itemize}
  \end{question}
\end{frame}

\begin{frame}
  \begin{question}
    \begin{itemize}
      \item Vilken vana tänker du prova och varför? Vad kommer att vara 
        utmaningen med detta?
      \item Har du något knep för att se till att du inte glömmer din nya vana?
    \end{itemize}
  \end{question}
\end{frame}

\begin{frame}
  \begin{question}
    \begin{itemize}
      \item Är prokrastinering under studierna annorlunda från prokrastinering 
        i yrkeslivet? Risker, konsekvenser.
    \end{itemize}
  \end{question}
\end{frame}

\begin{frame}
  \begin{question}
    \begin{itemize}
      \item Kan/bör man sträva efter att någon utifrån (chef, lärare, kollegor) 
        strukturerar arbetet eller måste man själv hitta metoder som fungerar 
        för en själv?
    \end{itemize}
  \end{question}
\end{frame}

\begin{frame}
  \begin{question}
    \begin{itemize}
      \item Hur hänger stress och prokrastinering ihop? Skapar prokrastinering 
        stress och/eller är stress en bra mekanism för att bryta en 
        prokrastineringssituation?
    \end{itemize}
  \end{question}
\end{frame}

\begin{frame}
  \begin{question}
    \begin{itemize}
      \item När är prokrastinering bra?
    \end{itemize}
  \end{question}
\end{frame}
